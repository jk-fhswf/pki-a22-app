\documentclass{article}


\usepackage[T1]{fontenc} 
\usepackage[ngerman]{babel}
\usepackage{graphicx}
\usepackage{subcaption}
\usepackage{hyperref}
\hypersetup{
	colorlinks   = true,
	citecolor    = blue
}

\newcommand*{\figref}[2][]{%
	\hyperref[{fig:#2}]{%
		Abbildung~\ref*{fig:#2}%
		\ifx\\#1\\%
		\else
		\,#1%
		\fi
	}%
}

\usepackage[
backend=biber,
style=alphabetic,
]{biblatex}
\addbibresource{literatur.bib}


\usepackage{listings}


\usepackage{xcolor}
\definecolor{codegreen}{rgb}{0,0.6,0}
\definecolor{codegray}{rgb}{0.5,0.5,0.5}
\definecolor{codeorange}{rgb}{1,0.49,0}
\definecolor{backcolour}{rgb}{0.95,0.95,0.96}


\lstdefinestyle{mystyle}{
	backgroundcolor=\color{backcolour},   
	commentstyle=\color{codegray},
	keywordstyle=\color{codeorange},
	numberstyle=\tiny\color{codegray},
	stringstyle=\color{codegreen},
	basicstyle=\ttfamily\footnotesize,
	breakatwhitespace=false,         
	breaklines=true,                 
	captionpos=b,                    
	keepspaces=true,                 
	numbers=left,                    
	numbersep=5pt,                  
	showspaces=false,                
	showstringspaces=false,
	showtabs=false,                  
	tabsize=2,
	xleftmargin=10pt,
}

\lstset{style=mystyle}
	
\begin{document}

\title{Programmierung für KI \\ Projektausarbeitung - Haar Cascades}
\author{Onur Yilmaz, Alexander Fuchs, Johannes-Peter Kübert, Peter Spanke}
\date{\today}
\maketitle
\newpage
	\tableofcontents
	\vspace{2cm} %Add a 2cm space

\newpage
\textbf{[Ausarbeitung - Onur]}
\section*{Einleitung}
Im Rahmen der Veranstaltung - \textit{Programmierung für KI}, haben wir das Thema \textbf{Haar Cascades}, als gemeinsam zu bearbeitendes Projekt erhalten. Haar Cascades beschreiben ein Verfahren bzw. einen Algorithmus, welches bestimmte Merkmale oder Muster in Bildern oder Videos erkennen kann.
\\ \\
Ziel dieses Projektes ist es nun ein solches Verfahren zu implementieren und eine passende graphische Benutzeroberfläche in Python bereitzustellen. Nach einer kurzen theoretischen Einführung in das Thema, teilt sich die Arbeit in zwei Teile auf. 
Der erste Teil beschäftigt sich mit der Erstellung einer graphischen Benutzeroberfläche mit Hilfe von \textbf{Tkinter} und der zweite Teil wiederum mit \textbf{streamlit}.
\\ \\
Schrittweise werden wir die einzelnen Python Codes hier darstellen und kennzeichnen, welche Codes genau, welchem Projektteilnehmer zuzuordnen sind. Für die Funktionalitäten der Programmcodes sind auf die Kommentare des beigefügten Programm Codes hingewiesen.

\section{Einführung - Haar Cascades}

Wie bereits oben erwähnt, sind Haar Cascades \footnote{ Rapid object detection using a boosted cascade of simple features und Viola, Jones: Robust Real-time Object Detection, IJCV 2001, zurück zu führen auf \textit{Paul Viola} und \textit{Michael Jones}} ein Verfahren zur Erkennung von Objekten in Bildern und Videos. Genauer handelt es sich hierbei um einen Klassifikationsalgorithmus, welches Beispielbilder trainiert, die entweder das zu erkennende Muster bzw. Objekt enthalten oder nicht. Anschließend kann der Haar Cascade dann auf neue Bilder oder Videos angewendet werden und versuchen, das gesuchte Muster oder Objekt darin wieder zu erkennen (siehe \cite{first} und \cite{second}).

\subsection{Funktionsweise: Haar-like Features}
Aus den Bildern der Eingangsdaten, werden für den Klassifikationsalgorithmuss nun die richtigen Merkmale (im engl. \textit{Feature}) extrahiert und trainiert. 
Hierbei benötigt man eine relativ große Trainingsmenge an positiven und negativen Graustufenbildern.\footnote{Genauer handelt es sich hierbei um normierte Graustufenbilder} 
\newline \\
Für die Merkmals-Extraktion werden nun rechteckige Bereiche (siehe \figref{haar1}) an einem Bild abgetastet. Hierfür gibt es sogenannte Kantenmerkmale ((1) und (2)), Linienmerkmale (3) und vier-rechteckige Merkmale (4).
\newpage
\begin{figure}[h!]
	\centering
	\includegraphics[width=0.6\linewidth]{haar1}
	\caption{Haar-like Features}
	\label{fig:haar1}
\end{figure}
\ \\
Um z.B. die Augen einer Person zu detektieren, stellen wir zunächst einmal fest, dass die Augen öfters dunkler sind, als andere Regionen im Gesicht. Hierzu verwenden wir dann das Feature (2) aus \figref{haar1}.
\begin{figure}[h!]
	\centering
	\begin{subfigure}{.5\textwidth}
		\centering
		\includegraphics[width=0.9\linewidth]{irina_sheyk}
		\label{fig:sub1}
	\end{subfigure}%
	\begin{subfigure}{.5\textwidth}
		\centering
		\includegraphics[width=0.9\linewidth]{irina_haar_cascade}
		\label{fig:sub2}
	\end{subfigure}
	\caption{Anwendung eines Haar-like Features}
	\label{fig:test}
\end{figure}
\ \\
Für alle Pixelwerte, z.B. normierte Helligkeitswerte, eines hellen bzw. dunklen Rechtecks wird der Durchschnitt gebildet und darauf hin die Werte der hellen und dunklen Rechtecke voneinander abgezogen \cite{drei}. \\ \\
Um zu ermitteln, welche Haar-like Features mit Form und Größe nun am
aussagekräftigsten sind,  gibt es den sogenannten
\textbf{Adaboost}-Algorithmus\footnote{AdaBoost (adaptives Boosting) ist ein Algorithmus für das Ensemble-Lernen, der für die Klassifikation oder Regression verwendet werden kann} 
Es wählt aus den vielen möglichen Merkmalen diejenigen aus, die die beste Unterscheidung zwischen einer Menge an Positiv- und Negativbeispielen von Mustern liefern und trainiert gleichzeitig den Klassifikator \cite{drei}.
\newpage
\subsection{Umsetzung in OpenCV}
Für unseren Anwendungsfall ist es mit Hilfe von OpenCV nicht notwendig einen eigenen Haar Cascade  Algorithmus anzulernen. Wir laden die XML-Datei mit den zugehörigen \textit{Haar-like-Features}, die wir verwenden möchten. Diese XML-Dateien enthalten dann anschließend Informationen darüber, wie die Features aussehen und wie sie auf Bilder angewendet werden sollen. 
\ \\ 
\section{Implementation in Tkinter}
\subsection{Standardaufbau einer Tkinter-App}
\textbf{[Onur]}
\begin{lstlisting}[language=Python, label = {lst:code}, mathescape= true]
import tkinter as tk

def druecke_knopf():
	label.config(text="Button gedrueckt!")

root = tk.Tk()
root.geometry("1200x800") 
root.title('Projekt Gruppe a2-2 - Thema: Bilderkennung Haar-Cascades')

button = tk.Button(root, text="Drueck mich", command=druecke_knopf)
button.pack()

label = tk.Label(root, text="Hallo Welt!")
label.pack()

root.mainloop()

\end{lstlisting}
Der obige Code erzeugt die recht simple Tkinter-Applikation mit einem Button und einem Label. Es sei hier auf \cite{vier} verwiesen.
\begin{figure}[h!]
	\centering
	\includegraphics[width=0.6\linewidth]{foto1}
	\caption{Einfache Tkinter App }
	\label{fig:Tkinter App - Auszug}
\end{figure}
\ \\ 
Zunächst einmal müssen wir Tkinter-Bibliothek importieren, welche essentiell für unsere App ist. 
\\ \\ 
Das Hauptfester wird mit dem Befehl
\begin{lstlisting}[language=Python]
	root = tk.Tk()
\end{lstlisting}
erzeugt und mit 
\begin{lstlisting}[language=Python]
	root.mainloop()
\end{lstlisting}
'beendet'. \\ \\ Letzteres trifft jedoch in Wirklichkeit nicht ganz zu, da wir uns in einer Endlosschleife befinden, welche nur dafür zuständig ist, darauf zu warten, bis der Benutzer auf etwas klickt \cite{vier}. Wenn der Benutzer schließlich dann auf einen Button klickt, wird ein Ereignis bzw. eine Funktion durchgeführt und anschließend wieder in die Endlosschleife versetzt.
\\ \\
Darüber hinaus können wir die Geometrie und Titel der App anpassen (siehe Zeile 7 und 8). 
\\ \\
In der obigen Anwendung haben wir einen Button und ein Label Modul aus der Tkinter-Bibliothek eingebaut. Die Referenzierung zu der Funktion erfolgt dann anschließend durch die \textit{command-Option}. Die Methode \textit{.pack()} von Tkinter kümmert sich um eine automatische Anordnung von Widgets, sprich Button, Labels, etc. im Fenster (root) zu verwalten.
\ \\ \\
\subsection{Haar-Cascade - Tkinter-App}
Um den Code möglichst lesbar zu gestalten, teilen wir im Folgenden, diesen in mehreren kleinen Blöcke auf.
\subsubsection{Packages}
\textbf{[Alle]}
\begin{lstlisting}[language=Python, label = {lst:code}, mathescape= true]
import tkinter as tk
from tkinter import ttk
from tkinter import filedialog
import numpy as np
import cv2
from PIL import Image, ImageTk
import random
import os

from pki_a22_app.utils.file_loader import get_classifiers

path_haarcascade = "resources/haarcascades/haarcascade_"


classifier_list = get_classifiers()
\end{lstlisting}

%Hier schreibe auf bzgl. den wichtigsten import 
% und füge hier die requirement.txt ein

\subsubsection{Initialisierung und Geometrie}
\textbf{[Alle]}
\begin{lstlisting}[language=Python]
root = tk.Tk()
root.title('Projekt Gruppe a2-2 - Thema: Bilderkennung Haar-Cascades')


screen_width = root.winfo_screenwidth()
screen_height = root.winfo_screenheight()


if screen_width/screen_height < 1.8:
	window_width = int(screen_width * 0.8)
else: 
	window_width = int(screen_height * (screen_width/2/screen_height)*0.8)
	window_height = int(screen_height * 0.8)

img_max_width = int(window_width/2-75)
img_max_height = int(window_height-300) 


center_x = int(screen_width/2 - window_width / 2)
center_y = int(screen_height/2 - window_height / 2)



root.geometry(f'{window_width}x{window_height}+{center_x}+{center_y}')
root.resizable(False, False)
\end{lstlisting}

% Hier legen wir die Größe der Tkinter App fest
% Außerdem haben wir hier eine angepasste Version der Fenster Höhe/Breite für verschiedene OS

\subsubsection{Bild öffnen und Klassifizierer anwenden}
\textbf{[Peter]}
\begin{lstlisting}[language=Python]
def file_open():
	global input_image
	global output_image
	filename = filedialog.askopenfilename(filetypes=(("jpg files", "*.jpg"),("png files", "*.png")))
	input_image = Image.open(filename)
	width, height = input_image.size
	aspect_ratio = width / height
	if aspect_ratio > 1:
		new_width = img_max_width
		new_height = int(new_width / aspect_ratio)
	else:  
	new_height = img_max_height
	new_width = int(new_height * aspect_ratio)
	input_image = input_image.resize((new_width,new_height), Image.ANTIALIAS)
	input_image_tk = ImageTk.PhotoImage(input_image)
	input_img_label.config(image=input_image_tk)
	input_img_label.image = input_image_tk

	output_image = Image.open(filename)
	output_image = output_image.resize((new_width,new_height), Image.ANTIALIAS)
	output_image_tk = ImageTk.PhotoImage(output_image)
	output_img_label.config(image=output_image_tk)
	output_img_label.image = output_image_tk
\end{lstlisting}

% Funktionen um ein beliebiges Bild im png oder jpg Format zu öffnen
% (aspect_ratio) --> Anzeige des Bildes erfolgt im richtigen Seitenverhältnis, erneut
% angepasst an der Fensterbreite und Höhe
\ \\
\textbf{[Peter und Alexander]}
\begin{lstlisting}[language=Python]
def img_change(classifier):
	global input_image
	global output_image
	cascade = cv2.CascadeClassifier(path_haarcascade + classifier + ".xml")
	output_image_cv = np.array(output_image.convert('RGB'))
	output_image_cv_gray = cv2.cvtColor(output_image_cv, cv2.COLOR_BGR2GRAY)
	cascade_results = cascade.detectMultiScale(output_image_cv_gray, scaleFactor=s1.get_val(), minNeighbors = s2.get_val(), minSize=(s3.get_val(), s3.get_val())) 
	iterations = 0

	if len(cascade_results) > 0:
		color = (random.randint(0,255),random.randint(0,255),random.randint(0,255))
		for (x,y,w,h) in cascade_results:
			cv2.rectangle(output_image_cv,(x,y),(x+w,y+h),(color),2)
			roi_gray = output_image_cv_gray[y:y+h, x:x+w]
			roi_color = output_image_cv[y:y+h, x:x+w]
		output_image = Image.fromarray(output_image_cv) 
		output_img_label.image.paste(output_image)
	else:
		flag = False
		for i1 in np.arange (1.5, 0.9, -0.1):
			for i2 in range (6, 2, -1):
				for i3 in range (50, 9, -10):                 
					s1.s.set(i1)
					s2.s.set(i2)
					s3.s.set(i3)
					iterations = iterations + 1
					cascade_results = cascade.detectMultiScale(output_image_cv_gray, scaleFactor=s1.get_val(), minNeighbors = s2.get_val(), minSize=(s3.get_val(), s3.get_val()))
					if len(cascade_results) > 0:
						color = (random.randint(0,255),random.randint(0,255),random.randint(0,255))
						for (x,y,w,h) in cascade_results:
							cv2.rectangle(output_image_cv,(x,y),(x+w,y+h),(color),2)
							roi_gray = output_image_cv_gray[y:y+h, x:x+w]
							roi_color = output_image_cv[y:y+h, x:x+w]
							output_image = Image.fromarray(output_image_cv)
							output_img_label.image.paste(output_image)
							flag = True
							break
					if flag: break
				if flag: break
			if len(cascade_results) == 0:
				print(f"Kein Ergebnis nach {iterations} Iterationen")          
\end{lstlisting}

\textbf{[Peter]}
\begin{lstlisting}[language=Python]
def output_image_restart():
	global input_image
	global output_image
	output_image = input_image
	output_img_label.image.paste(output_image)
\end{lstlisting}
\ \\ 
\textbf{[Peter]}
\subsubsection{Slider und Button}
\begin{lstlisting}[language=Python]
class slider:    
	def __init__(self,name,x_pos=0,y_pos=0,scale_from=0,scale_to=100,typ=int):
		self.name = name
		self.x_pos = x_pos
		self.y_pos = y_pos
		self.scale_from = scale_from
		self.scale_to = scale_to
		if typ==int: 
			self.val = tk.IntVar() 
		else: 
			self.val = tk.DoubleVar() 

		self.lbl = ttk.Label(root, text=name)
		self.lbl.place(x=self.x_pos,y=self.y_pos)

		self.s = ttk.Scale(root, from_=self.scale_from, to=self.scale_to, 	orient="horizontal",command=self.slider_change,variable=self.val)
		self.s.place(x=self.x_pos+100,y=self.y_pos)

		self.v = ttk.Label(root,text=f"{self.get_val():.1f}")
		self.v.place(x=self.x_pos+210,y=self.y_pos)
	def get_val(self):
		return self.val.get()
	def slider_change(self, event):
		self.v.configure(text=f"{self.get_val():.1f}")
\end{lstlisting}

% da nicht absehbar war, wie viele Slider am Ende benutzt wurde hier eine Klasse verwendet.
\ \\ 
\textbf{[Alexander]}
\begin{lstlisting}[language=Python]
def blur_rectangle(classifier):
	global input_image
	global output_image
	cascade = cv2.CascadeClassifier(path_haarcascade + classifier + ".xml")
	output_image_cv = np.array(output_image.convert('RGB'))
	output_image_cv_gray = cv2.cvtColor(output_image_cv, cv2.COLOR_BGR2GRAY)
	cascade_results = cascade.detectMultiScale(output_image_cv_gray, scaleFactor=s1.get_val(), minNeighbors = s2.get_val(), minSize=(s3.get_val(), s3.get_val()))
		if len(cascade_results) > 0:
			for (x,y,w,h) in cascade_results:
				face = output_image_cv[y:y+h, x:x+w]
				face = cv2.GaussianBlur(face, (23, 23), 30)
				output_image_cv[y:y+h, x:x+w] = face

			output_image = Image.fromarray(output_image_cv)
			output_img_label.image.paste(output_image)

def save_jpg():
	file_path = filedialog.asksaveasfilename(initialfile="output_image", filetypes=(("jpg files", "*.jpg"),("png files", "*.png")), defaultextension=".jpeg")
	if file_path:
		output_image.save(file_path)
\end{lstlisting}
\ \\
\textbf{[Peter]}
\begin{lstlisting}[language=Python]
tk.Button(root, text="Bild aus Datei oeffnen", command=file_open).place(x=window_width/2-window_width/4,y=150)
tk.Button(root, text="Classifier anwenden", command=lambda: img_change(dropdown.get())).place(x=window_width/2+window_width/4,y=150)
tk.Button(root, text="==>", command=output_image_restart).place(x=window_width/2-12.5,y=window_height/2)
tk.Button(root, text="Weichzeichnen",command=lambda: blur_rectangle(dropdown.get())).place(x=window_width/2+window_width/4+122,y=150)
tk.Button(root, text="Bild speichern",command=save_jpg).place(x=window_width/2+window_width/4+220,y=150)

dropdown = tk.StringVar(root)
dropdown.set(classifier_list[2])
dropdown_label = tk.OptionMenu(root, dropdown, *classifier_list)
dropdown_label.place(x=window_width/2-75,y=20)


xpos_slider_window = window_width/2 -75
ypos_slider_window = 60
s1 = slider("ScaleFactor",xpos_slider_window,ypos_slider_window,1.01,1.5,float)
s1.s.set(1.1)
s2 = slider("MinNeighbors",xpos_slider_window,ypos_slider_window+30,3,6)
s2.s.set(4)
s3 = slider("minSize",xpos_slider_window,ypos_slider_window+60,10,50)
s3.s.set(30)


input_img_label = ttk.Label(root)
input_img_label.place(x=25,y=200)

output_img_label = ttk.Label(root)
output_img_label.place(x=window_width/2+50,y=200)

fh_logo = Image.open("resources/images/Logo.jpg")
fh_logo = fh_logo.resize((300,100), Image.ANTIALIAS)
fh_logo_tk = ImageTk.PhotoImage(fh_logo)
ttk.Label(root, image=fh_logo_tk).place(x=0,y=0)

root.mainloop()	
\end{lstlisting}

\section{Implementierung mit Streamlit } 
\textbf{[Johannes]} \\

Eine weitere Implementierung des Haar Cascade Klassifizierers wurde 
mit dem Dashboard Framework \textbf{Streamlit} als \textbf{Web-Anwendung} umgesetzt. Das Dashboard enthält
zum einen einen Bereich mit unterschiedlichen Parametern für den 
Haar Cascade Klassifizierer, und zum anderen einen Bereich, in dem 
verschieden Bild-Quellen ausgewählt werden können. Dazu gehört
neben eigenen oder vorinstallierten Bildern auch die Möglichkeit, ein Video (Siehe \figref{streamlit1})
oder das eigene Webcam-Bild zu verwenden. Für die Übertragung von Video und Webcam Bildern wurde die 
\href{https://github.com/whitphx/streamlit-webrtc}{WebRTC-Erweiterung} für Streamlit verwendet .

\begin{figure}[h]
 	\includegraphics[scale=0.3]{../images/streamlit/video_example.png}
	\caption{Streamlit Dashboard mit Haar Cascade Klassifizierer}
	\label{fig:streamlit1}
\end{figure}

Das Dashboard kann auch als \href{https://jk-fhswf-pki-a22-app-app-codcuk.streamlit.app/}{Demo-Anwendung}
auf der Streamlit-Cloud ausprobiert werden. 

\subsection{Eingabequellen}
Die Anwendung unterstützt diese bereits angesprochenen Eingabemöglichkeiten:

\begin{description}
	\item[Upload]\hfill \\
	Hier kann ein einzelnes Bild hochgeladen und verwendet werden
	\item[Dataset]\hfill \\
	Unter diesem Punkt können mehrere vorkonfigurierte Bilder auf einmal verwendet werden
	\item[Video]\hfill \\
	Bei dieser Option wird ein vorkonfiguriertes Video abgespielt, auf dem der ausgewählte 
	Klassifizierer angewandt wird
	\item[Webcam]\hfill \\
	Hier besteht die Möglichkeit, den Klassifizierer auf die eigene Webcam-Bilder anzuwenden
\end{description}

\subsection{Optionen}
Die verwendeten Optionen zur Konfiguration des Haar Cascade Klassifizierers \textbf{Scale Factor},
\textbf{Min Neighbors} und \textbf{Min Size} wurden bereits in den vorherigen Abschnitten im Rahmen der
Tkinter Anwendung beschrieben und können hier nachgelesen werden: \cite{hackaday}

\subsection{Code}
Der Programmcode für die Implementierung des Streamlit-Dashboards ist auf die folgenden Quelldateien verteilt:

\begin{lstlisting}[language=Python]
# Hauptanwendung und Einstiegspunkt
app.py
# Funktion zur Ausfuehrung des eigentlichen Klassifizierers
pki_a22_app/haarcascades/haarcascades.py
# Interface fuer die unterschiedlichen Eingabequellen
pki_a22_app/dashboard/sources.py
# Eingabequelle mit einem einzelnen hochladbarem Bild
pki_a22_app/dashboard/source_upload.py
# Eingabequelle mit Auswahl eines Datensatzes mit mehreren Bildern
pki_a22_app/dashboard/source_dataset.py
# Eingabequelle zum Abspielen eines voreingestellten Videos
pki_a22_app/dashboard/source_video.py
# Eingabequelle bei der die eigene Webcam verwendet wird
pki_a22_app/dashboard/source_webcam.py
# Utility Modul, das vor allem Funktionen fuer Datei-Operationen enthaelt
pki_a22_app/utils/file_loader.py
\end{lstlisting}
\bigbreak

Im folgenden Abschnitt wird der Programmcode im Detail aufgelistet. Die Dokumentation zu den einzelnen Abschnitten befindet sich im Code selbst: 
\bigbreak

\lstinputlisting[caption=app.py, language=Python]{../../app.py}
\lstinputlisting[caption=haarcascades/haarcascades.py, language=Python]{../../pki_a22_app/haarcascades/haarcascades.py}
\newpage
\begin{figure}
	\centering
	\includegraphics[scale=0.5]{input_sources.png}
	\caption{UML Diagramm für unterschiedliche Source-Typen}
\end{figure}
\lstinputlisting[caption=dashboard/sources.py, language=Python]{../../pki_a22_app/dashboard/sources.py}
\lstinputlisting[caption=dashboard/source upload.py, language=Python]{../../pki_a22_app/dashboard/source_upload.py}
\lstinputlisting[caption=dashboard/source dataset.py, language=Python]{../../pki_a22_app/dashboard/source_dataset.py}
\lstinputlisting[caption=dashboard/source video.py, language=Python]{../../pki_a22_app/dashboard/source_video.py}
\lstinputlisting[caption=dashboard/source webcam.py, language=Python]{../../pki_a22_app/dashboard/source_webcam.py}
\lstinputlisting[caption=utils/file loader.py, language=Python]{../../pki_a22_app/utils/file_loader.py}


\newpage

\section{Organisation}

\textbf{[Alle]} \\

Besonders zu Beginn war es sehr hilfreich, eine Kollaborations-Plattform 
für das Projekt einzurichten, über die die unterschiedlichsten Informationen
organisatorischer Natur ausgetauscht werden konnten. 
Wir haben uns für den Einsatz eines Miro-Boardes (\figref{miro}) entschieden. das wir 
unter anderem nutzten für:

\begin{itemize}
	\item Terminfindung
	\item Pinnwand
	\item Brainstorming
	\item Zeitplanung
	\item Wireframes
\end{itemize}

\begin{figure}[ht]
	\includegraphics[scale=0.25]{../images/miro_board.png}
	\caption{Miro-Board für organisatorische Aufgaben}
	\label{fig:miro}
\end{figure}

\newpage

\section{Continuous Integration}

\textbf{[Johannes]} \\

Um gemeinsam am Programmcode arbeiten zu können, haben wir uns ein Git-basiertes 
Codeverwaltungs-Tool entschieden. Hierfür haben wir auf GitHub ein eigenes
\href{https://github.com/jk-fhswf/pki-a22-app}{Git-Repository} (\figref{git}) eingerichtet.


\begin{figure}[!h]
 	\includegraphics[scale=0.25]{../images/ci_git.png}
 	\caption{Git-Repository auf GitHub}
	\label{fig:git}	
\end{figure}

Zur Qualitätssicherung haben wir uns dafür entschieden, für den main-Branch
Pull-Requests zu verwenden und die Erfolgreiche Ausführung eines GitHub-Actions-
Workflows für Unit-Tests vorauszusetzen (\figref{git_qa}). Eine Beispiellauf des
Workflows ist in \figref{pytest} zu sehen.

\begin{figure}[!h]
 	\includegraphics[scale=0.25]{../images/ci_protection.png}
 	\caption{Konfiguration von QA-Optionen}
	\label{fig:git_qa}
\end{figure}


\begin{figure}[!h]
 	\includegraphics[scale=0.25]{../images/ci_actions.png}
 	\caption{Beispiel-Ausführung von Tests mit pytest}
	\label{fig:pytest}
\end{figure}



\newpage
 
\printbibliography
\end{document}
