\section{Implementierung mit Streamlit } 
\textbf{[Johannes]} \\

Eine weitere Implementierung des Haar Cascade Klassifizierers wurde 
mit dem Dashboard Framework \textbf{Streamlit} als \textbf{Web-Anwendung} umgesetzt. Das Dashboard enthält
zum einen einen Bereich mit unterschiedlichen Parametern für den 
Haar Cascade Klassifizierer, und zum anderen einen Bereich, in dem 
verschieden Bild-Quellen ausgewählt werden können. Dazu gehört
neben eigenen oder vorinstallierten Bildern auch die Möglichkeit, ein Video (Siehe \figref{streamlit1})
oder das eigene Webcam-Bild zu verwenden. Für die Übertragung von Video und Webcam Bildern wurde die 
\href{https://github.com/whitphx/streamlit-webrtc}{WebRTC-Erweiterung} für Streamlit verwendet .

\begin{figure}[h]
 	\includegraphics[scale=0.3]{../images/streamlit/video_example.png}
	\caption{Streamlit Dashboard mit Haar Cascade Klassifizierer}
	\label{fig:streamlit1}
\end{figure}

Das Dashboard kann auch als \href{https://jk-fhswf-pki-a22-app-app-codcuk.streamlit.app/}{Demo-Anwendung}
auf der Streamlit-Cloud ausprobiert werden. 

\subsection{Eingabequellen}
Die Anwendung unterstützt diese bereits angesprochenen Eingabemöglichkeiten:

\begin{description}
	\item[Upload]\hfill \\
	Hier kann ein einzelnes Bild hochgeladen und verwendet werden
	\item[Dataset]\hfill \\
	Unter diesem Punkt können mehrere vorkonfigurierte Bilder auf einmal verwendet werden
	\item[Video]\hfill \\
	Bei dieser Option wird ein vorkonfiguriertes Video abgespielt, auf dem der ausgewählte 
	Klassifizierer angewandt wird
	\item[Webcam]\hfill \\
	Hier besteht die Möglichkeit, den Klassifizierer auf die eigene Webcam-Bilder anzuwenden
\end{description}

\subsection{Optionen}
Die verwendeten Optionen zur Konfiguration des Haar Cascade Klassifizierers \textbf{Scale Factor},
\textbf{Min Neighbors} und \textbf{Min Size} wurden bereits in den vorherigen Abschnitten im Rahmen der
Tkinter Anwendung beschrieben und können hier nachgelesen werden: \cite{hackaday}

\subsection{Code}
Der Programmcode für die Implementierung des Streamlit-Dashboards ist auf die folgenden Quelldateien verteilt:

\begin{lstlisting}[language=Python]
# Hauptanwendung und Einstiegspunkt
app.py
# Funktion zur Ausfuehrung des eigentlichen Klassifizierers
pki_a22_app/haarcascades/haarcascades.py
# Interface fuer die unterschiedlichen Eingabequellen
pki_a22_app/dashboard/sources.py
# Eingabequelle mit einem einzelnen hochladbarem Bild
pki_a22_app/dashboard/source_upload.py
# Eingabequelle mit Auswahl eines Datensatzes mit mehreren Bildern
pki_a22_app/dashboard/source_dataset.py
# Eingabequelle zum Abspielen eines voreingestellten Videos
pki_a22_app/dashboard/source_video.py
# Eingabequelle bei der die eigene Webcam verwendet wird
pki_a22_app/dashboard/source_webcam.py
# Utility Modul, das vor allem Funktionen fuer Datei-Operationen enthaelt
pki_a22_app/utils/file_loader.py
\end{lstlisting}
\bigbreak

Im folgenden Abschnitt wird der Programmcode im Detail aufgelistet. Die Dokumentation zu den einzelnen Abschnitten befindet sich im Code selbst: 
\bigbreak

\lstinputlisting[caption=app.py, language=Python]{../../app.py}
\lstinputlisting[caption=haarcascades/haarcascades.py, language=Python]{../../pki_a22_app/haarcascades/haarcascades.py}
\newpage
\begin{figure}
	\centering
	\includegraphics[scale=0.5]{input_sources.png}
	\caption{UML Diagramm für unterschiedliche Source-Typen}
\end{figure}
\lstinputlisting[caption=dashboard/sources.py, language=Python]{../../pki_a22_app/dashboard/sources.py}
\lstinputlisting[caption=dashboard/source upload.py, language=Python]{../../pki_a22_app/dashboard/source_upload.py}
\lstinputlisting[caption=dashboard/source dataset.py, language=Python]{../../pki_a22_app/dashboard/source_dataset.py}
\lstinputlisting[caption=dashboard/source video.py, language=Python]{../../pki_a22_app/dashboard/source_video.py}
\lstinputlisting[caption=dashboard/source webcam.py, language=Python]{../../pki_a22_app/dashboard/source_webcam.py}
\lstinputlisting[caption=utils/file loader.py, language=Python]{../../pki_a22_app/utils/file_loader.py}


\newpage

\section{Organisation}

\textbf{[Alle]} \\

Besonders zu Beginn war es sehr hilfreich, eine Kollaborations-Plattform 
für das Projekt einzurichten, über die die unterschiedlichsten Informationen
organisatorischer Natur ausgetauscht werden konnten. 
Wir haben uns für den Einsatz eines Miro-Boardes (\figref{miro}) entschieden. das wir 
unter anderem nutzten für:

\begin{itemize}
	\item Terminfindung
	\item Pinnwand
	\item Brainstorming
	\item Zeitplanung
	\item Wireframes
\end{itemize}

\begin{figure}[ht]
	\includegraphics[scale=0.25]{../images/miro_board.png}
	\caption{Miro-Board für organisatorische Aufgaben}
	\label{fig:miro}
\end{figure}

\newpage

\section{Continuous Integration}

\textbf{[Johannes]} \\

Um gemeinsam am Programmcode arbeiten zu können, haben wir uns ein Git-basiertes 
Codeverwaltungs-Tool entschieden. Hierfür haben wir auf GitHub ein eigenes
\href{https://github.com/jk-fhswf/pki-a22-app}{Git-Repository} (\figref{git}) eingerichtet.


\begin{figure}[!h]
 	\includegraphics[scale=0.25]{../images/ci_git.png}
 	\caption{Git-Repository auf GitHub}
	\label{fig:git}	
\end{figure}

Zur Qualitätssicherung haben wir uns dafür entschieden, für den main-Branch
Pull-Requests zu verwenden und die Erfolgreiche Ausführung eines GitHub-Actions-
Workflows für Unit-Tests vorauszusetzen (\figref{git_qa}). Eine Beispiellauf des
Workflows ist in \figref{pytest} zu sehen.

\begin{figure}[!h]
 	\includegraphics[scale=0.25]{../images/ci_protection.png}
 	\caption{Konfiguration von QA-Optionen}
	\label{fig:git_qa}
\end{figure}


\begin{figure}[!h]
 	\includegraphics[scale=0.25]{../images/ci_actions.png}
 	\caption{Beispiel-Ausführung von Tests mit pytest}
	\label{fig:pytest}
\end{figure}

